\section*{Gedanken über das \href{http://one.laptop.org/}{\textit{One Laptop per Child Projekt [1]}}}
\hypertarget{olpcnewyear}{}
\label{olpcnewyear}
\NewsAuthor{Christoph Derndofer}

\textbf{\href{http://christoph-d.blogspot.co.at/}{\textit{Christoph Derndorfer [2]}} ist Mitgründer von \href{http://olpc.at/?setlang=de}{\textit{OLPC Austria [3]}}, und forscht, reist, programmiert und \href{http://www.olpcnews.com/}{\textit{bloggt [4]}} auf englisch und spanisch über das \href{http://one.laptop.org/}{\textit{One Laptop per Child (OLPC) Projekt [1]}} und über Möglichkeiten die Bildungssituation in Entwicklungsländern zu verbessern.} \\

\begin{figure}
\includegraphics[width=\linewidth]{olpcnewyear/olpc-christoph.jpg} \\
\caption{Christoph Derndorfer. Bildrechte: [2]}
\end{figure}

Hintergrund: \\

Ziel des \href{http://one.laptop.org/}{\textit{One Laptop per Child Projekts}} war es einen billigen 100 US\$ Computer für jedes Kind in Entwicklungsländern zu entwickeln. Obwohl dieses Ziel nie ganz erreicht wurde führten die damit einhergehenden technologischen Durchbrüche zur Kategorie der Billig-Netbooks und günstigen Tablet-Computer und Entwicklung von \href{https://en.wikipedia.org/wiki/Sugar_(desktop_environment)}{\textit{open source Lernsoftware [5]}}. Das \href{http://one.laptop.org/}{\textit{One Laptop per Child Projekt}} wurde vom charismatischen (und umstrittenen) MIT Professor \href{https://de.wikipedia.org/wiki/Nicholas_Negroponte}{\textit{Nicholas Negroponte [6]}} gegründet und hatte den bisher größten Erfolg in Peru und Uruguay. In Österreich gab es zwischen 2008 und 2012 eine \href{http://www.fuzo-archiv.at/artikel/1503081v2}{\textit{Probeklassen in der Steiermark die den OLPC XO Laptop [7]}} regulär im Unterricht einsetzte. \\

Übersetzung von \href{http://spielend-programmieren.at}{\textit{Horst JENS [8]}}. Originaltitel: \href{http://www.olpcnews.com/commentary/olpc_news/happy_new_year_from_olpc_news.html}{\textit{Happy new year from OLPC News [9]}} \\

(Beginn der Übersetzung) \\

Während ich hier sitze und durch die Postings von 2013 durchblättere anstatt mich für die Silvesterparty vorzubereiten bekomme ich einen Gedanken nicht aus meinem Kopf: 2013 war ein seltsames Jahr für One Laptop per Child. \\

Ich meine nicht dass es ein ereignisloses Jahr war. Aber die geringen Anzahl an Postings -nur zwei Dutzend im letzten Jahr- zeigt doch deutlich dass nicht allzuviel passiert ist was mich zum schreiben inspiriert hätte. Die Wahrheit ist, dass vieles von dem was passiert ist mich eher deprimiert hat. \\

Auch wenn das viele Leute innerhalb der globalen OLPC Community nicht wahrhaben wollen: Es wurden sehr offensichtlich dass OLPC heute eine andere Organisation ist als sie 2012 und davor war. Natürlich bedeutete 'anders' nicht automatische 'schlechter'. Ich persönlich kann mich aber weder sehr für das neue XO Tablet (auf das sich die OLPC Association in Miami 2013 großteils konzentriert hat) begeistern noch finde ich es besonders wertvoll. \\

Die große Herausforderung, die mich antreibt, bleibt es herauszufinden wie man Informations- und Kommunikationstechnologien in das Bildungssystem von Entwicklungsländern integrieren kann. Wir haben viel darüber gelernt was funktioniert und was nicht funktioniert seit OLPC 2005 gegründet wurde. Und ich traue mich zu behaupten, dass die Welt dank der Bemühungen der Organisation, ihrer Angestellten und weltweiten Unterstützer ein besserer Ort geworden ist. \\

Allerdings habe ich das Gefühl dass 2013 gezeigt hat dass wir an einem toten Punkt angelangt sind. \\

OLPC Association als Organisation kümmert sich nicht länger um die richtigen Fragen, produziert keine relevanten Antworten und hat an personellen Kapazitäten verloren, welche sie die letzten Jahre über innehatte. \\

Als olpc Community und größer werdendes Ökosystem haben wir noch nicht herausgefunden wie wir mit diesen Änderungen umgehen sollen. \\

Klar, immer noch arbeiten Leute an interessanten technischen Lösungen von denen einige sicherlich später sehr wertvoll werden. Aber davon abgesehen ist nicht klar wer die schweren Herausforderungen lösen wird (oder zumindest versuchen wird sie zu lösen) welche ich die sechs Kriterien einer erfolgreichen Integration von Informationstechnologie in den Bildungsbereich von Entwicklungsländern nennen möchte:

\begin{itemize}
\item Infrastruktur
\item Wartung
\item Unterrichtsmaterialien (Content)
\item Einbindung der Communities
\item Lehrerausbildung
\item Evaluierung (Erfolgskontrolle)
\end{itemize}

Wenn wir erfolgreich sein wollen, müssen wir uns diesen Herausforderungen stellen und nicht darauf hoffen das jemand anderer dies für uns tun wird. Denn das ist der Geist der OLPC von Anfang an angetrieben hat. \\

Die Frage ist -egal ob in Österreich oder in Zambia- nicht länger \textit{ob} wir Informationstechnologie im Unterricht verwenden sollen oder nicht. Die Frage ist welche Technologie wir verwenden sollen und -wichtiger noch- \textit{wie} wir sie verwenden wollen. \\

Wenn wir bei diesen Fragen mitreden wollen anstatt zuzuschauen wie sich die Klassenzimmer der Welt mit ungeeignetem Geräten und veralteten pädagogischen Konzepten füllen - dann sollten wir uns besser mehr anstrengen und auf die wichtigen Fragen konzentrieren. \\

Anderenfalls finden wir uns in ein paar Jahren abseits der Entwicklung stehend und darüber jammernd welche Chancen wir versäumt haben und mit Vorwürfen und Fehlersuche beschäftigt. Ich weiß nicht wie es Euch damit geht aber ich habe andere Pläne für 2018. \\

Mit diesen Gedanken im Kopf wünsche ich Euch allen ein glückliches neues Jahr und freue mich auf 2014! \\

(Ende der Übersetzung )

%\subsection*{Fachbegriffe:}

\subsection*{Spenden:}

OLPC (Austria) freut sich über jegliche Unterstützung und Spenden. Konktakt: \textit{office@olpc.at}. \\

\subsection*{Download, Feedback:}
\textbf{R.I.S.-Journal}, Ausgabe 001: \\
\href{http://spielend-programmieren.at/de:ris:001}{spielend-programmieren.at/de:ris:001}\\
\textbf{Download} Ordner, verschiedene Formate: \href{http://spielend-programmieren.at/risjournal/001/olpcnewyear}{\texttt{spielend-programmieren.at/\\risjournal/001/olpcnewyear}} \\
\textbf{Feedback} \Letter\ \texttt{christoph@olpcnews.com} \\

\subsection*{Lizenz, Quellen:}
\begin{wrapfigure}{l}{2.0cm}
\includegraphics[width=2cm]{olpcnewyear/ccbysa88x31.png}
\end{wrapfigure}
Dieses Material steht unter der Creative-Commons-Lizenz Namensnennung - Weitergabe unter gleichen Bedingungen 4.0 International. Um eine Kopie dieser Lizenz zu sehen, besuchen Sie \url{http://creativecommons.org/licenses/by-sa/4.0/deed.de}.

\textbf{Quellen:} \\
{[}1{]} \href{http://one.laptop.org/}{one.laptop.org} \\
{[}2{]} \href{http://christoph-d.blogspot.co.at/}{christoph-d.blogspot.co.at} \\
{[}3{]} \href{http://olpc.at}{olpc.at} \\
{[}4{]} \href{http://www.olpcnews.com/}{olpcnews.com} \\
{[}5{]} \href{https://en.wikipedia.org/wiki/Sugar_(desktop_environment)}{http://goo.gl/1SE4Xm} \\
{[}6{]} \href{https://de.wikipedia.org/wiki/Nicholas_Negroponte}{Nikolas Negroponte} \\
{[}7{]} \href{http://www.fuzo-archiv.at/artikel/1503081v2}{fuzo-archiv.at/artikel/1503081v2} \\
{[}8{]} \href{http://spielend-programmieren.at}{spielend-programmieren.at} \\
{[}9{]} \href{http://www.olpcnews.com/commentary/olpc_news/happy_new_year_from_olpc_news.html}{goo.gl/uvojYk}
