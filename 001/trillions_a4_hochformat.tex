%%%%%%%%%%%%%%%%%%%%%%%%%%%%%%%%%%%%%%%%%
% Short Three-Column Newsletter
% LaTeX Template
% Version 1.0 (11/9/13)
%
% Original author:
% Frits Wenneker (http://www.howtotex.com) 
% With extensive modifications by:
% Vel (vel@latextemplates.com)
% 
% This template has been downloaded from:
% http://www.LaTeXTemplates.com
%
% License:
% CC BY-NC-SA 3.0 (http://creativecommons.org/licenses/by-nc-sa/3.0/)
%
%%%%%%%%%%%%%%%%%%%%%%%%%%%%%%%%%%%%%%%%%

%----------------------------------------------------------------------------------------
%	PACKAGES AND DOCUMENT CONFIGURATIONS
%----------------------------------------------------------------------------------------

\documentclass[10pt,a4paper,ngerman,twoside]{article} % Paper type (a4paper, usletter or legal) and font size (10, 11 or 12)

%\setlength\topmargin{-80mm} % Top margin
\setlength\topmargin{-48pt} % Top margin
\setlength\headheight{0pt} % Header height
\setlength\textwidth{7.0in} % Text width
\setlength\textheight{9.5in} % Text height
\setlength\oddsidemargin{-30pt} % Left margin
\setlength\evensidemargin{-30pt} % Left margin (even pages) - only relevant with 'twoside' article option
%\setlength\inner{4cm}
%\setlenfth\outer{2cm}
%\usepackage{geometry}
%\geometry{bindingoffset=20mm}
%\setlength\bindingoffset{2cm}

\usepackage{charter} % Charter font for main content

\frenchspacing % Reduces space after periods to make text more compact for a three-column layout
\usepackage{babel}
\usepackage[utf8]{inputenc}
\usepackage{graphicx} % Required for including images
\usepackage{amssymb} % Math packages
\usepackage{amsmath} 
\usepackage{multicol} % Required for the three-column layout of the document
\usepackage{url} % Clickable links
\usepackage{enumitem} % Reduces the amount of space within and between lists with [noitemsep,nolistsep]
\usepackage{marvosym} % Required for the use of symbols
\usepackage{wrapfig} % Allows wrapping text around figures
%\usepackage[T1]{fontenc} % Use 8-bit encoding that has 256 glyphs
\usepackage{datetime} % Required for defining a custom date style
\newdateformat{mydate}{\monthname[\THEMONTH] \THEYEAR} % Set a custom date format
\usepackage[pdfpagemode=FullScreen, colorlinks=false]{hyperref} % Link colors and PDF behavior in Acrobat
\usepackage{fancyhdr} % Required to define custom headers/footers
\usepackage{hyperref} % funktioniert nicht ?
\pagestyle{fancy} % Enables the custom headers/footers for all pages following this

%-----------------------------------------------------------
% Header and footer
\lfoot{\footnotesize % Left footer containing newsletter contact information
%\begin{wrapfigure}{l}{2.0cm}
%\includegraphics[width=2cm]{ccbysa88x31.png} 
%\end{wrapfigure}
R.I.S. Journal Ausgabe 001, Jänner 2014: \textbf{R}emix, \textbf{I}mprove, \textbf{S}hare. Das freie, creativ-commons lizensierte Journal.  \\
\Mundus\ Download und andere Formate: \href{http://spielend-programmieren.at/de:ris:start}{\texttt{spielend-programmieren.at/de:ris:start}} \quad
%\Telefon\ (000) 111-1111 \quad
\Letter\ \href{mailto:horst.jens@spielend-programmieren.at}{horst.jens@spielend-programmieren.at}
}

\cfoot{} % Empty center footer

\rfoot{\footnotesize ~\\ Seite \thepage} % Right footer - page counter

\renewcommand{\headrulewidth}{0.0pt} % No horizontal rule for the header
\renewcommand{\footrulewidth}{0.4pt} % Horizontal rule separating the footer from the document
%-----------------------------------------------------------

%-----------------------------------------------------------
% Define separators
\newcommand{\HorRule}[1]{\noindent\rule{\linewidth}{#1}} % Creates a horizontal rule
\newcommand{\SepRule}{\noindent	% Creates a shorter separator rule
\begin{center}
\rule{250pt}{1pt} % Page width and rule width
\end{center}
}
\newcommand{\Trenner}{\noindent
\begin{center}
\rule{100pt}{1pt}
\end{center}
}
%-----------------------------------------------------------

%-----------------------------------------------------------
% Define title and article styles
\newcommand{\NewsletterName}[1]{ % Newsletter title
\begin{center}
\Huge \usefont{T1}{fvs}{b}{n} % Use the Bera Sans Bold font
#1
\end{center}	
\par \normalsize \normalfont}

\newcommand{\JournalIssue}[1]{ % Date and issue number at the top of the newsletter
%\hfill \textsc{\mydate \today, No #1} % Right-aligned date and issue number
\hfill \textsc{Jänner 2014, Ausgabe 001}
\par \normalsize \normalfont}

\newcommand{\NewsItem}[1]{ % News item title
\usefont{T1}{fvs}{n}{n} % Use the Bera Sans Normal font
\vspace{24pt}\large #1\vspace{3pt} % Print the title with space around it in a larger font size
\par \normalsize \normalfont}

\newcommand{\NewsAuthor}[1]{ % Author name under the item title
\hfill von \textsc{#1} \vspace{20pt} % Right-aligned author name in small caps with space after it
\par \normalfont}		

%----------------------------------------------------------------------------------------
%	TITLE
%----------------------------------------------------------------------------------------

\begin{document}

\JournalIssue{1} % Issue number
\NewsletterName{R.I.S. Journal} % Newsletter title
%\begin{center}
%\textbf{R}emix \textbf{I}mprove \textbf{S}hare - das freie Journal für Open Source Education
%\end{center}
\noindent\HorRule{3pt} \\[-0.75\baselineskip] % Thick horizontal rule
\HorRule{1pt} % Thin horizontal rule



%\setlength{\columnsep}{16pt} % Uncomment to manually change the white space between columns
% % Begin the three-column layout

%----------------------------------------------------------------------------------------
%	OTHER NEWS
%----------------------------------------------------------------------------------------
%-----------------------------------------------------------
%
%-----------------------------------------------------------
%RIS-Journal Titel (Titelgrafik hier einfügen)

\NewsItem{}
\section*{Wie man Billion\"ar wird}
\label{trillions}
\NewsAuthor{David CAIN}

\textbf{In diesem Blogposting vom Jänner 2011 (!) erklärt \href{http://www.raptitude.com/contact/}{\textit{David Cain [1]}} den Unterschied zwischen viel Geld, wirklich viel Geld und viel zu viel Geld. Das Originalposting erschien auf dem \href{http://www.raptitude.com/2011/01/how-to-make-trillions-of-dollars}{\textit{Raptitude.com Blog [2]}}. Originaltitel: How to make trillions of dollars.}

\begin{figure}
\includegraphics[width=0.9\linewidth]{trillions/trillions-davidcain.jpg} 
\caption{David Cain. Bildrechte: [1]}
\end{figure}

Übersetzung von Gerald PINK und Horst JENS mit freundlicher Genehmigung des Autors. Anmerkung: Die Namen für große Zahlen sind im Amerikanischen etwas anders als im Deutschen (ein beliebter Übersetzungsfehler). Zur Übersicht: \\
\texttt{
Eine Million  (engl.: million)\\
ist (eine 1 und 6 Nullen):\\
\textbf{1.000.000}\\
Eine Millarde (engl.: billion)
sind 1.000 Millionen:\\
\textbf{1.000.000.000}\\
Eine Billion  (engl.: trillion) \\
sind 1.000 Milliarden: \\
\textbf{1.000.000.000.000}
}

(Beginn der Übersetzung) \\


Zu Beginn möchte ich klarstellen, dass ich das nicht empfehle. Ich erkläre diese Strategie nur um zu informieren, damit du verstehst in welcher Umgebung du arbeitest, und damit du für dich persönlich gute Entscheidungen treffen kannst, darüber, wie du dein Geld verdienst und wie du mit Geld umgehst.

Ich ermuntere Dich dazu ein Millionär zu werden, wenn dich das interessiert. Falls du Milliardär werden möchtest finde ich das zwar verdächtig, aber ich möchte dich nicht vorverurteilen. Wenn du Billionär werden willst: das ist nur etwas für die wirklich Bösen. Dann können wir leider keine Freunde sein.

Richtig großes Geld steckt nicht im Produzieren von Gütern sondern im Schaffen von Kunden. Ein einzelner, lebenslang treuer Kunde der sein Leben so verbringt, wie du es willst, ist Millionen wert. Millionen solcher Kunden zu schaffen, das ist der Weg zum Billionär.

Du kannst Millionen machen mit dem Verkauf eines großartigen Produkts, welches die Leute brauchen; aber du machst Milliarden und Billionen damit, ganze Nationen und Völker dazu zu erziehen, auf jedes WehWeh, auf jeden Wunsch, auf jede flüchtige Begierde oder Angst mit KAUFEN zu reagieren.

Du brauchst etwas Geld um die Billionärs-Strategie in Schwung zu bekommen. Das geht nicht mit indischen 5 Dollar/Stunde Assistenten. Aber, wie gesagt, ich  möchte sowieso nicht dass du Billionär wirst, weil das nicht so gut für den Rest von uns ist .

Um in Schwung zu kommen braucht es einen Kulturwandel. Kleinunternehmer kriegen das nie hin, aber wenn du etwas 'altes Geld' (Vermögen) geerbt hast (irgendwer hat das immer) dann bist du in der Position um Politiker, Zeitungen und Trendsetter zu beeinflussen, und dann kannst du ziemlich schnell definieren, was in Zukunft als 'normal' gilt.

Nach dem Ende des 2.Weltkriegs entwickelten ein paar privilegierte Amerikaner eine brilliante Formel um eine unvorstellbar große Ökonomie zu erschaffen:


American retail analyst \href{https://en.wikipedia.org/wiki/Victor_Lebow}{Victor Lebow} [3]:
\begin{quote}
\textit{
[Unsere Wirtschaft] verlangt dass wir ein Dasein als Konsument führen, dass wir aus dem Kauf von und dem Gebrauch von Gütern ein Ritual machen, dass wir unsere spirituelle und persönliche Erfüllung im Konsum suchen. Um sozialen Status zu messen, Prestige, soziale Akzeptanz ... das findet man in unserem Konsumverhalten [...] Wir müssen Dinge konsumieren, verbrennen, verbrauchen, ersetzten, wegschmeißen in einem immer schnelleren Temo. Wir müssen die Menschen essen, trinken, anziehen, Auto fahren lassen mit immer komplizierteren - und deshalb immer teurerem -  Konsumgütern.}
\end{quote} 

(Danke an Leserin Anna für den Hinweis)

Das ist sehr High-level-Marketing, und es hat den Großteil der entwickelten Welt um uns herum geformt.

Durch Benutzung von vor allem Massenmedien (Fernsehen) haben es High-Level Marketing Männer geschafft eine Nation zu erschaffen deren Menschen typischerweise:

\begin{itemize}
\item fast die ganze Zeit arbeiten
\item sich jede Nacht zu Hause stundenlang Werbung reinziehen
\item müde sind, \href{http://www.raptitude.com/2010/10/being-healthy-is-not-normal/}{ungesund} [4] leben und mit ihrem Leben unzufrieden sind
\item auf Langeweile, Unzufriedenheit oder Ängstlichkeit nur auf genau eine Weise reagieren: Dinge KAUFEN
\item Geld zur Verfügung haben aber keine erfüllendere Arbeit finden aus Angst davor die Sozialversicherung zu verlieren
\item Gesundheitsprobleme für sich selber erzeugen, welche mit Tabletten behandelt werden.
\item \href{http://www.raptitude.com/2011/01/i-dont-want-stuff-any-more-only-things/}{weitaus mehr Dinge besitzen} [5] als sie benutzten, und trotzdem glauben dass sie hat nicht genug (Dinge) haben
\item sehr leicht abgelenkt sind von ihrer ungesunden Lebenssituation und -Kultur durch ständige Nachrichten (breaking news) sowie TV-Tratsch über Prominente
\item ständig selbst überzeugt sind dass genau jetzt nicht der richtige Zeitpunkt ist um den Lebensstil zu ändern
\item gerne Sachen kaufen die ein Jahr später kaputt gehen und von dem niemand weiß wie man sie repariert
\item gelernt haben durch die Medienkultur der Schuldzuweisung, dass der Schlüssel zur privaten und öffentlichen Problemlösung darin liegt herauszufinden welche Leute man hassen muss
\end{itemize}

Im Rest der entwickelten Welt ist es nicht viel anders. Billionäre und ihre Angestellten arbeiten rund um die Uhr daran diese Gewohnheiten per TV in die Wohnzimmer der Menschheit zu pumpen. Große Wirtschaftsmacht, große Bevölkerung, globaler Einfluss.

Gesunde Menschen - Menschen die wissen wie man mit Enttäuschungen umgeht, Menschen die nicht an 'Wunderlösungen' (Magic Bullets) glauben, Menschen welche nicht ungehemmt TV schauen, Menschen welche Erfüllung finden durch die  Dinge im Leben die kein Geld kosten - solche Menschen sind schlechte (poor) Konsumenten. Die High-Level-Marketing Männer haben eine Kultur produziert die genau das Gegenteil propagiert.

Heutzutage wirst du gedrängt, aus allen Richtungen, ungesund und unerfüllt zu werden oder zu bleiben, denn dann willst du mehr KAUFEN. Ich möchte dir nicht Angst machen, aber dies ist der primäre Zweck des matt schimmernden Kastens in deinem Wohnzimmer - dich auffzufordern zu schlechter ( gerade noch nicht Konsum hemmender) Gesundheit, genereller Wehleidigkeit und einem unbewussten Reflex: Geld auszugeben. 

Dieser Text ist nicht ein Aufruf dazu den nächsten \textbf{Wal-Mart} nieder zu brennen. Der Anti-Kaptialismus, das  ist doch etwas für kleine Kinder.

Ich rate dir auch nicht dazu den Fernseher aus dem Fenster zu schmeissen. Obwohl das ein Akt mit einem hohen  \href{https://de.wikipedia.org/wiki/Return_on_Investment}{\textbf{Return of Investement}} wäre, falls dich der Gedanke interessiert. 

Ich rate nichtmal dazu einen Hass auf 'den Mann' (der diese Strategie erfunden hat) zu schüren. Es wäre eine Untertreibung zu sagen dass wäre so wie den falschen Baum anbellen. Da draußen ist nichts - keine Ideologie, Methodologie, Verhalten, Person oder Idee - welche es wert wäre zu hassen. 'Der Mann' macht was er macht weil er es nicht besser weiß - er ist selbst abhängig und weiß wenig vom Leben.

Deine Lebensfähigkeiten (Skills) können besser sein, so dass sie dich davor schützen zerstörerischen und selbst-beschädigenden Drängen nachzugeben. Solchen Drängen wie -unter anderem-  z.B. dem Drang eine Kultur zu schaffen der schlechten Gesundheit, eine Kultur der Angst, eine Kultur der Sucht nur damit du exklusive Sensationen erlebst wie das Tragen von 10.000 Dollar Unterwäsche. Oder an der Börse mit Firmen zocken so als wären sie Spielkarten. 

Ein wirklich armer Mann ist derjendige der es nicht zusammenbringt ein komfortables Leben zu führen ohne dafür Millionen Menschen zu verarmen um leichter an deren Geld zu kommen. Er kann da nicht raus - er ist geldsüchtig.

Also hasse nicht 'den Mann', das macht dich nur schwach und lenkt dich ab davon was du erreichen kannst wenn du nicht seinen Ratschlägen folgst. Hass hat einen wirklich lausigen Return Of Investment, zumindest was die Lebensqualität betrifft. Wenn du überkochst vor Wut und Dampf ablassen musst, betreibe Sport. Richte deinen Zorn weg von anderen Menschen und Tieren. Wenn du genug Dampf abgelassen hast, richte deine Energie auf die Verbesserung deiner Lebensgewohnheiten und -Fähigkeiten.

Ich rede gar nicht über spezielle Fähigkeiten wie z.B. Reifen wechseln, Hemden bügeln oder Waschräume öffnen ohne die Türklinke zu berühren, obwohl sowas nützlich ist.

Ich rede über die Fundamente: Ein selbstbestimmter Mensch zu sein.

Kreativität. Neugier. Widerstand gegen Ablenkung. Geduld mit Mitmenschen.

Und um das alles zu ermöglichen: Selbstsicherheit - unbeugsamer Willen die Verantwortung für das eigene Leben zu übernehmen. Egal wer aller daran mitgewirkt hat dass dein Leben so ist wie es ist.

Kultiviere diese Fähigkeiten in dir selbst und in Anderen. Wenn diese Lebenseinstellung normal wird dann werden die Lifestyle-Tipps von Plakatwerbung, Sonderangeboten und CNN -die bewährte Billionärs-Strategie- nicht mehr wirken. Bei niemandem mehr.

Sogar von einem ohnmächtigem Startpunkt aus - ein mieses Zimmer in einer vergessenen Ecke der Gesellschaft welches geradezu erschaffen wurde dazu dich krank und impotent zu machen - Diese Lebensfähigkeiten werden mehr für dich tun als jede 'Anti'-Haltung die du dir vorstellen kannst. Das System zu hassen ist Amerikas Lieblingsbeschäftigung in der Vergangenheit gewesen. Es fühlt sich gut an, einmal begonnen ist es schwer damit aufzuhören, und es führt dich exakt nirgendwo hin. So etwa wie \textbf{Doritos} essen.

Es geht nicht um wir gegen die, es geht um wir für uns. (englisches Wortspiel: us for U.S.)

Obwohl 'der Mann' in vielerlei Hinsicht unreif ist, so hat er doch eines: sehr viel Geld. Seine Macht wirst du überall spüren. 


Er hat deinen \href{http://www.raptitude.com/2010/07/your-lifestyle-has-already-been-designed/}{Lifestyle designt}  [6], beginnend mit den Jahren wo du einer Arbeit nachgehst die nichts mit deinen Leidenschaften zu tun hat; von der Autowerbung die du anstarrst während du in einer Restauranttoilette bist, über deinen DVD-Player der genau 30 Tage länger lebt als die  Gewährleistungspflicht vorschreibt, bis zum ausgetretenen Pfad zwischen deiner Mikrowelle und deiner Fernseh-Couch.


Die Billionärs-Strategien die du erlebst sind so schrecklich weil sie schon als so normal aufgefasst werden. Du wirst merken wie sie sich in dein Leben hineinwinden. Nicht nur explizit - durch Zeitschriften und Werbung - sondern auch implizit, in der Art wie deine Mitmenschen, deine Lieben, von dir erwarten zu leben.

Du wirst ungewöhnlich werden müssen. Ich hoffe du traust dich.


(Ende der Übersetzung)

\subsection*{Fachbegriffe:}

~~~\textbf{Wal-Mart} Ein amerikanische Supermarktkette, gehört den erzkonservativen Koch-Brüdern (u.a. Sponsoren der Tea-Party)

\textbf{Doritos} amerikanische Kartoffelchips (Markenbezeichnung). 

\href{https://de.wikipedia.org/wiki/Return_on_Investment}{\textbf{Return of Investment}} abgekürzt ROI: Grob gesagt der Gewinn dividiert durch das eingesetzte Kapital: Wie schnell man eine Investition wieder 'hineinverdient' hat. 

\subsection*{Download, Feedback:}
\footnotesize{
Download: Ordner \texttt{trillions} \Mundus\ \href{http://spielend-programmieren.at/risjournal/001}{spielend-programmieren.at/risjournal/001}\\
Startseite:\\
\href{http://spielend-programmieren.at/de:ris:001}{spielend-programmieren.at/de:ris:001}\\ 
\Letter\: horst.jens@spielend-programmieren.at\\
\Letter\: @DavidDCain (Twitter)\\}
\normalsize 

\subsection*{Lizenz, Quellen:}

\subsubsection*{Übersetzung (deutsch):}

\begin{wrapfigure}{l}{2.0cm}
\includegraphics[width=2cm]{trillions/ccbysa88x31.png}
\end{wrapfigure}
Dieses Material steht unter der Creative-Commons-Lizenz Namensnennung - Weitergabe unter gleichen Bedingungen 4.0 International. Um eine Kopie dieser Lizenz zu sehen, besuchen Sie \url{http://creativecommons.org/licenses/by-sa/4.0/deed.de}.

\subsubsection*{Original (englisch):}

\begin{wrapfigure}{l}{2.0cm}
\includegraphics[width=2cm]{trillions/ccbynd88x31.png}
\end{wrapfigure}
Der Originaltext ist lizensiert unter einer Creative-Commons-Lizenz Namensnennung - NoDerivatives 4.0 International. Um eine Kopie dieser Lizenz zu sehen, besuchen Sie \url{http://creativecommons.org/licenses/by-nd/4.0/}


\textbf{Quellen:} \\
{[}1{]} \href{https://twitter.com/DavidDCain}{David Cain} \\
bzw: \href{http://www.raptitude.com/contact/}{raptitude.com/contact} \\
{[}2{]} \href{http://www.raptitude.com/2011/01/how-to-make-trillions-of-dollars/}{goo.gl/g7bSCF} \\
{[}3{]} \href{https://en.wikipedia.org/wiki/Victor\_Lebow}{en.wikipedia.org/wiki/Victor\_Lebow} \\
{[}4{]} \href{http://www.raptitude.com/2010/10/being-healthy-is-not-normal/}{http://goo.gl/Q2F9cM} \\
{[}5{]} \href{http://www.raptitude.com/2011/01/i-dont-want-stuff-any-more-only-things/}{http://goo.gl/S5xbPH} \\
{[}6{]} \href{http://www.raptitude.com/2010/07/your-lifestyle-has-already-been-designed/}{http://goo.gl/AL1Rcy} 


\SepRule{}
%-----------------------------------------------------------
\end{document}
