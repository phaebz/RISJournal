http://opensource.com/education/14/1/teaching-kids-linux

Vier Linux Distributionen für Kinder

Posted 14 Jan 2014 by Aseem Sharma http://www.google.com/url?q=http%3A%2F%2Fopensource.com%2Fusers%2Faseem-sharma&sa=D&sntz=1&usg=AFQjCNFLUbGasDRceUQBTQ-z9hho-xLlbA


Ich kann das neugierige Leuchten in den Augen meiner sechs Jahre alten Nichte Shuchi sehen, wenn sie an einem Handy herumspielt, die Glotze mit der Fernbedienung verstellt oder irgend ein anderes elektrisches Gerät aus kreative Weise auseinander nimmt. Wie viele Kinder ihres Alters ist Shuchi sehr experimentierfreudig.


Diese Neugierde erreicht ihren Höhepunkt immer dann wenn sie vor meinem Laptop (oder dem Laptop ihres Vaters) sitzt. Häufig aber beobachte ich jedoch, dass sie sich in Anwendungen verirrt, die nur auf Erwachsene zugeschnitten sind. Ein Betriebssystem für Erwachsene und die Logik dahinter erscheinen vielen Kindern monströs. Diese Anwendungen übersteigen das Verständnis kleiner Kinder und sind nicht gerade der ideale (und spielerische) Einstieg in den Umgang mit dem Computer. Außerdem sind Laptops und Tablet-PCs für Erwachsene keine geeignete Lern-umgebung für Kinder (egal ob jüngere oder ältere), die gerade erst in die Computerwelt einsteigen. Und nicht zuletzt kann auch die Vorstellung ein Kind mit einem Computer mit Internetanbindung alleine zu lassen für die Eltern sehr beängstigend sein.


Da ich selbst auch nur ein großes Kind bin und seit mehr als vier Jahren auch ein Open Source Software Enthusiast, liebe ich es mit verschiedenen Softwarelösungen herum zu experimentieren und zu probieren. Betreffend das Problem ein ideales System für meine junge Nichte zu finden und einzurichten, fiel mir auf, dass dass die Open Source Linuxgemeinde spezielle Betriebssysteme und Umgebungen für Kinder geschaffen hat. Ein solches System aufzusetzen ist eine Kleinigkeit.


Warum Kinder Linux lernen sollten.


In dieser Phase meines Lebens habe ich mir eine abschließende Meinung gebildet, dass Kinder schon früh mit der Macht von Linux konfrontiert werden sollten. Zwei der Gründe dafür sind...


Die Zukunft des Computerwesens


Kürzlich las ich einen excellenten Artikel von Stu Jarvis, Ein Jahr Linux Desktop an der Westcliff High School, (siehe RIS-Journal 001, westcliff), in dem Malcom Moore auf eine Frage mit folgendem Satz antwortet:  

\begin{quote}\textit{
Hier ist eine Studie, die besagt, dass im Jahr 2000 Windows auf 97%% aller Computergeräte installiert war, dass aber heute mit Smartphones und Tablet-PCs, usw. Windows nur noch auf 20%% der Computergeräte läuft und in der Welt der Supercomputer regiert Linux unangefochten. Wir sind auf Wissenschaft und Entwicklung spezialisiert und wollen das unsere Studenten weiterhin so große Taten vollbringen, wie das nächste Google zu schaffen oder das Universum im CERN kollabieren zu lassen. In diesen Umfeldern müssen sie sich selbstverständlich mit Linux auseinandersetzen.}\end{quote}

Linux treibt einige der komplexesten Systeme der Welt an. Für jemanden der auch nur entfernt an einer Karriere im Technologiesektor interessiert ist, sind Linuxkenntnisse ein absoluter Vorteil. Ganz nebenbei bemerkt hat der Einsatz überall massiv zugenommen. Man bedenke:

* Linux betreibt internationale Raumstationen


* Linux treibt neue Technolgie in Autos wie denen von Tesla oder Cadillac an


* Linux betreibt Luftfahrtkontrollsysteme


* Google, Facebook, Twitter, alle bauen auf  Linux


* 9 von 10 Supercomputern auf der Welt nutzen Linux

Es gibt einen vernünftigen Grund warum Initiative wie One Laptop per Child, das meiner Meinung nach eine der wichtigsten Programme der Gegenwart ist, welches versucht die digitale Kluft zu überbrücken,Linux-basierte Systeme nutzt.


Anpassungsfähigkeit und Vielfältigkeit


Lernen im frühen Alter kann am besten durch eine Umgebung gefördert werden , die das eingenständige Erforschen fördert. Es gibt kein anderes Betriebssystem, das eine solche Vielfalt liefert und dem Nutzer so viele Möglichkeiten gibt das System auf bestimmte Bedürfnisse hin anzupassen wie Linux. So wie es Spielzeug und Kleidung für Kinder gibt, hat die Linuxgemeinde spezielle Betriebssysteme entwickelt, die Kindern eine spaßige Lernumgebung bieten. Ich glaube das es wichtig ist eine Umgebung zu schaffen die Kindern ein Gefühl für das wunderbare gibt, um ihre Neugierde anzufeuern.


Programme um Kindern Linux beizubringen


Es gibt viele verschiedene Arbeitsumgebungen, die die Linuxgemeinde extra für Kinder geschaffen hat und ich habe noch nicht alle ausprobiert. Doch unter denen, die ich bereits getestet habe, sollte es möglich sein eine geeignete Lösung dabei sein um den meisten Kinder Linux und das Computerwesen beizubringen.


Qimo


http://www.qimo4kids.com/



Qimo for kids ist eine Distribution, die auf Ubuntu basiert und speziell für Kinder entwickelt wurde. Diese Variant hat etliche Lernanwendungen für Kinder ab 3 Jahren vorinstalliert. Dabei ist GCompris, eine perfekte Sammlung für Kinder von 3 - 10. Sie besteht aus über 100 Lernspielen, die die Grundlagen der Computernutzung vermitteln, aber auch das Lesen, Kunstgeschichte, die Uhrzeit und Zeichnen. Außerdem beinhaltet sie auch Childs Play, eine Sammlung von Spielen zum Gedächtnistrining.

Was mit an dieser Distribution mit am besten gefällt ist, dass sie die XFCE-Arbeitsumgebung einsetzt, die sehr niedrige Hardwareanforderungen hat und damit auch auf älteren Rechnern gut funktioniert.. Damit ist es fast schon lächerlich einfach einen alten Laptop oder Desktop-Rechner um zu funktionieren.Wir hatten so einen alten Rechner zuhause und Qimo hat ihn zu neuem Leben erweckt. Dieses System war, wegen seiner kinderfreundlichen Zeichentrick-Arbeitsfläche und seiner Auswahl an Lernsoftware, meine erste Wahl für meine junge Nichte.

Sugar

https://www.sugarlabs.org/

Sugar wurde für das One Laptop per Child Programm entwickelt. Es ist ein einfach zu benutzendes und Kinderfreundliches System. Kinder, die gerne entdecken, werden sich in dieser Arbeitsumgebung schnell zurechtfinden, selbst wenn sie noch nicht lesen können.

Die Sugar Labs:

Bei Information geht es um Substantive; beim Lernen um Verben. Die  Sugar Benutzeroberfläche, mit ihrer Abkehr von der traditionellen Form der Computer-Arbeitsfläche, ist der erste ernsthafte VErsuch eine Benutzeroberfläche zu schaffen, die auf den Prinzipien des kognitiven und sozialen Konstruktivismus basiert: Lernende sollen sich in echter Erforschung und Zusammenarbeit beteiligen. Es basiert auf drei ganz einfachen Prinzipien darüber was und Menschen ausmacht..


Ubermix

http://www.ubermix.org/

Ubermix wird in Schule ausgiebig genutzt. Das System wurde entwickelt um Nutzerdaten uns Software auf getrennten Partitionen  zu speichern. Daher kann man im Falle einer Computerstörung das System problemlos löschen und schnell neue  Kopien eines Systems erstellen. 

Der Ubermix Gründer, Jim Klein, im Opensource.com Interview:

Ubermix kommt mit einer fertig eingerichteten großen Auswahl an Programmen für. Bildung, produktives Arbeiten, Entwicklung, Programmierung, Internet und Mediendesign. Bildungsorientierte Programme wie Celestia, Stellarium, Scratch, VirtualLab Microscope, Geogebra, iGNUit, und Klavaro, sowie Lernspiele wie TuxMath, TuxTyping, gMult, und Numpty Physics bieten viele Möglichkeiten zum Lernen.

Alle Internetanwendungen die wir kennen und lieben gelernt haben, wie Firefox, Thunderbird, Chrome, Google Earth, und Skype sind da. Verbreitete Arbeitsanwendungen wie LibreOffice, NitroTasks, Planner Project Management, VYM (View Your Mind), und Zim Desktop Wiki dürfen auch nicht fehlen. Kinder die sich für das Gestalten interessieren werden GIMP, Inkscape, Scribus, Dia, Agave, und sogar TuxPaint für die jüngeren finden..Weitere Software wie Audacity, Openshot, Pencil, und ffDiaporama helfen das Medienangebot abzurunden. Diese und viele weitere machen Ubermix zu einer mächtigen Lern- und Kreativitätsplattform für Schüler.


Edubuntu

http://www.edubuntu.org/

Edubuntu, das eigentlich Ubuntu Education Edition heißt,  wurde in Zusammenarbeit mit Erziehern und Lehrern entwickelt. Es bettet eine Reihe von Lernprogrammen in eine geeignete Lernumgebung ein.Ein Vorteil ist ihr Zugang zum ganzen Ubuntu Software Repository.Lehrende haben diese Distribution bereits umfangreich in Schulen und anderen Bildungseinrichtungen verwendet um eine reichhaltige Lernumgebung für Lernende zur Verfügung zu stellen. Es ist ein großartiges System um älteren Kindern Linux beizubringen, da seine Lernkurve am Anfange etwas steiler sein kann als etwa bei Qimo oder Sugar.


